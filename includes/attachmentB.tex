\begin{lstlisting}[
  caption={Štruktúra kern\_ipc\_perm},
  label={lst:kernipcperm},
  language=c
]
struct kern_ipc_perm {
	spinlock_t	lock;
	bool		deleted;
	int		id;
	key_t		key;
	kuid_t		uid;
	kgid_t		gid;
	kuid_t		cuid;
	kgid_t		cgid;
	umode_t		mode;
	unsigned long	seq;
	void		*security;
	struct rhash_head khtnode;
	struct rcu_head rcu;
	refcount_t refcount;
} ____cacheline_aligned_in_smp __randomize_layout;
\end{lstlisting}
\begin{lstlisting}[
  caption={Štruktúra msqid64\_ds},
  label={lst:msqid64},
  language=c
]
struct msqid64_ds {
	struct ipc64_perm msg_perm;
	__kernel_time_t msg_stime;	/* last msgsnd time */
#if __BITS_PER_LONG != 64
	unsigned long	__unused1;
#endif
	__kernel_time_t msg_rtime;	/* last msgrcv time */
#if __BITS_PER_LONG != 64
	unsigned long	__unused2;
#endif
	__kernel_time_t msg_ctime;	/* last change time */
#if __BITS_PER_LONG != 64
	unsigned long	__unused3;
#endif
	__kernel_ulong_t msg_cbytes;	/* current number of bytes on queue */
	__kernel_ulong_t msg_qnum;	/* number of messages in queue */
	__kernel_ulong_t msg_qbytes;	/* max number of bytes on queue */
	__kernel_pid_t msg_lspid;	/* pid of last msgsnd */
	__kernel_pid_t msg_lrpid;	/* last receive pid */
	__kernel_ulong_t __unused4;
	__kernel_ulong_t __unused5;
};
\end{lstlisting}
\begin{lstlisting}[
  caption={Štruktúra msqid\_ds},
  label={lst:msqid},
  language=c
]
struct msqid_ds {
	struct ipc_perm msg_perm;
	struct msg *msg_first;		/* first message on queue,unused  */
	struct msg *msg_last;		/* last message in queue,unused */
	__kernel_time_t msg_stime;	/* last msgsnd time */
	__kernel_time_t msg_rtime;	/* last msgrcv time */
	__kernel_time_t msg_ctime;	/* last change time */
	unsigned long  msg_lcbytes;	/* Reuse junk fields for 32 bit */
	unsigned long  msg_lqbytes;	/* ditto */
	unsigned short msg_cbytes;	/* current number of bytes on queue */
	unsigned short msg_qnum;	/* number of messages in queue */
	unsigned short msg_qbytes;	/* max number of bytes on queue */
	__kernel_ipc_pid_t msg_lspid;	/* pid of last msgsnd */
	__kernel_ipc_pid_t msg_lrpid;	/* last receive pid */
};
\end{lstlisting}
\begin{lstlisting}[
  caption={Štruktúra msginfo},
  label={lst:msginfo},
  language=c
]
struct msginfo {
	int msgpool;
	int msgmap; 
	int msgmax; 
	int msgmnb; 
	int msgmni; 
	int msgssz; 
	int msgtql; 
	unsigned short  msgseg; 
};
\end{lstlisting}
\begin{lstlisting}[
  caption={Štruktúra semid\_ds},
  label={lst:semid},
  language=c
]
struct semid_ds {
	struct ipc_perm	sem_perm;		/* permissions .. see ipc.h */
	__kernel_time_t	sem_otime;		/* last semop time */
	__kernel_time_t	sem_ctime;		/* create/last semctl() time */
	struct sem	*sem_base;		/* ptr to first semaphore in array */
	struct sem_queue *sem_pending;		/* pending operations to be processed */
	struct sem_queue **sem_pending_last;	/* last pending operation */
	struct sem_undo	*undo;			/* undo requests on this array */
	unsigned short	sem_nsems;		/* no. of semaphores in array */
};
\end{lstlisting}
\begin{lstlisting}[
  caption={Štruktúra semid64\_ds},
  label={lst:semid64},
  language=c
]
struct semid64_ds {
	struct ipc64_perm sem_perm;	/* permissions .. see ipc.h */
	__kernel_time_t	sem_otime;	/* last semop time */
	__kernel_ulong_t __unused1;
	__kernel_time_t	sem_ctime;	/* last change time */
	__kernel_ulong_t __unused2;
	__kernel_ulong_t sem_nsems;	/* no. of semaphores in array */
	__kernel_ulong_t __unused3;
	__kernel_ulong_t __unused4;
};
\end{lstlisting}
\begin{lstlisting}[
  caption={Štruktúra semun},
  label={lst:semun},
  language=c
]
union semun {
	int val;			/* value for SETVAL */
	struct semid_ds __user *buf;	/* buffer for IPC_STAT & IPC_SET */
	unsigned short __user *array;	/* array for GETALL & SETALL */
	struct seminfo __user *__buf;	/* buffer for IPC_INFO */
	void __user *__pad;
};
\end{lstlisting}
\begin{lstlisting}[
  caption={Štruktúra shmid\_kernel},
  label={lst:shmid},
  language=c
]
struct shmid_kernel /* private to the kernel */
{	
	struct kern_ipc_perm	shm_perm;
	struct file		*shm_file;
	unsigned long		shm_nattch;
	unsigned long		shm_segsz;
	time64_t		shm_atim;
	time64_t		shm_dtim;
	time64_t		shm_ctim;
	pid_t			shm_cprid;
	pid_t			shm_lprid;
	struct user_struct	*mlock_user;

	/* The task created the shm object.  NULL if the task is dead. */
	struct task_struct	*shm_creator;
	struct list_head	shm_clist;	/* list by creator */
} __randomize_layout;
\end{lstlisting}
\begin{lstlisting}[
  caption={Ukážka použitia soketu},
  label={lst:server},
  language=c
]
#include <stdio.h>
#include <stdlib.h>
#include <errno.h>
#include <string.h>
#include <sys/types.h>
#include <sys/socket.h>
#include <sys/un.h>

#define SOCK_PATH "echo_socket"

int main(void)
{
    int s, s2, t, len;
    struct sockaddr_un local, remote;
    char str[100];

    if ((s = socket(AF_UNIX, SOCK_STREAM, 0)) == -1) {
        perror("socket");
        exit(1);
    }

    local.sun_family = AF_UNIX;
    strcpy(local.sun_path, SOCK_PATH);
    unlink(local.sun_path);
    len = strlen(local.sun_path) + sizeof(local.sun_family);
    if (bind(s, (struct sockaddr *)&local, len) == -1) {
        perror("bind");
        exit(1);
    }

    if (listen(s, 5) == -1) {
        perror("listen");
        exit(1);
    }

    for(;;) {
        int done, n;
        printf("Waiting for a connection...\n");
        t = sizeof(remote);
        if ((s2 = accept(s, (struct sockaddr *)&remote, &t)) == -1) {
            perror("accept");
            exit(1);
        }

        printf("Connected.\n");

        done = 0;
        do {
            n = recv(s2, str, 100, 0);
            if (n <= 0) {
                if (n < 0) perror("recv");
                done = 1;
            }

            if (!done) 
                if (send(s2, str, n, 0) < 0) {
                    perror("send");
                    done = 1;
                }
        } while (!done);

        close(s2);
    }

    return 0;
}
\end{lstlisting}
\begin{lstlisting}[
  caption={Ukážka použitia soketu na strane klienta},
  label={lst:client},
  language=c
]
#include <stdio.h>
#include <stdlib.h>
#include <errno.h>
#include <string.h>
#include <sys/types.h>
#include <sys/socket.h>
#include <sys/un.h>

#define SOCK_PATH "echo_socket"

int main(void)
{
    int s, t, len;
    struct sockaddr_un remote;
    char str[100];

    if ((s = socket(AF_UNIX, SOCK_STREAM, 0)) == -1) {
        perror("socket");
        exit(1);
    }

    printf("Trying to connect...\n");

    remote.sun_family = AF_UNIX;
    strcpy(remote.sun_path, SOCK_PATH);
    len = strlen(remote.sun_path) + sizeof(remote.sun_family);
    if (connect(s, (struct sockaddr *)&remote, len) == -1) {
        perror("connect");
        exit(1);
    }

    printf("Connected.\n");

    while(printf("> "), fgets(str, 100, stdin), !feof(stdin)) {
        if (send(s, str, strlen(str), 0) == -1) {
            perror("send");
            exit(1);
        }

        if ((t=recv(s, str, 100, 0)) > 0) {
            str[t] = '\0';
            printf("echo> %s", str);
        } else {
            if (t < 0) perror("recv");
            else printf("Server closed connection\n");
            exit(1);
        }
    }

    close(s);

    return 0;
}
\end{lstlisting}

\begin{lstlisting}[
  caption={K-objekt procesu},
  label={lst:kobjekt},
  language=c
]
struct process_kobject { /* was: m_proc_inf */

	pid_t pid, parent_pid, child_pid, sibling_pid;
	struct pid* pgrp;
	kuid_t uid, euid, suid, fsuid;
	kgid_t gid, egid, sgid, fsgid;
        char cmdline[128];

	kuid_t luid;
	kernel_cap_t ecap, icap, pcap;
	MEDUSA_SUBJECT_VARS;
	MEDUSA_OBJECT_VARS;
	__u32 user;
#ifdef CONFIG_MEDUSA_SYSCALL
	/* FIXME: this is wrong on non-i386 architectures */

		/* bitmap of syscalls, which are reported */
	unsigned char med_syscall[NR_syscalls / (sizeof(unsigned char) * 8)];
#endif
};
\end{lstlisting}

\begin{lstlisting}[
  caption={LSM hooky pre IPC},
  label={lst:hook},
  language=c
]
	//alokovanie
	LSM_HOOK_INIT(msg_msg_alloc_security, medusa_l1_msg_msg_alloc_security),
	LSM_HOOK_INIT(msg_msg_free_security, medusa_l1_msg_msg_free_security),
	LSM_HOOK_INIT(msg_queue_alloc_security, medusa_l1_msg_queue_alloc_security),
	LSM_HOOK_INIT(msg_queue_free_security, medusa_l1_msg_queue_free_security),
	LSM_HOOK_INIT(shm_alloc_security, medusa_l1_shm_alloc_security),
	LSM_HOOK_INIT(shm_free_security, medusa_l1_shm_free_security),
	LSM_HOOK_INIT(sem_alloc_security, medusa_l1_sem_alloc_security),
	LSM_HOOK_INIT(sem_free_security, medusa_l1_sem_free_security),

	LSM_HOOK_INIT(ipc_permission, medusa_l1_ipc_permission),
	LSM_HOOK_INIT(ipc_getsecid, medusa_l1_ipc_getsecid),	
	LSM_HOOK_INIT(msg_queue_associate, medusa_l1_msg_queue_associate),
	LSM_HOOK_INIT(msg_queue_msgctl, medusa_l1_msg_queue_msgctl),
	LSM_HOOK_INIT(msg_queue_msgsnd, medusa_l1_msg_queue_msgsnd),
	LSM_HOOK_INIT(msg_queue_msgrcv, medusa_l1_msg_queue_msgrcv),
	LSM_HOOK_INIT(shm_associate, medusa_l1_shm_associate),
	LSM_HOOK_INIT(shm_shmctl, medusa_l1_shm_shmctl),
	LSM_HOOK_INIT(shm_shmat, medusa_l1_shm_shmat),
	LSM_HOOK_INIT(sem_associate, medusa_l1_sem_associate),
	LSM_HOOK_INIT(sem_semctl, medusa_l1_sem_semctl),
	LSM_HOOK_INIT(sem_semop, medusa_l1_sem_semop),
\end{lstlisting}
\begin{lstlisting}[
  caption={Alokovanie bezpečnostnej štruktúry},
  label={lst:alloc},
  language=c
]
int medusa_l1_ipc_alloc_security(struct kern_ipc_perm *ipcp, unsigned int ipc_class)
{
	struct medusa_l1_ipc_s *med;

	med = (struct medusa_l1_ipc_s*) kmalloc(sizeof(struct medusa_l1_ipc_s), GFP_KERNEL);
	if (med == NULL)
		return -ENOMEM;

	med->ipc_class = ipc_class;
	ipcp->security = med;
	return 0;
}
\end{lstlisting}
\begin{lstlisting}[
  caption={Uvoľnenie pamäte bezpečnostnej štruktúry},
  label={lst:free},
  language=c
]
void medusa_l1_ipc_free_security(struct kern_ipc_perm *ipcp)
{
	struct medusa_l1_ipc_s *med;
	
	if(ipcp->security != NULL) {
		med = ipcp->security;
		ipcp->security = NULL;
		kfree(med);
	}
}
\end{lstlisting}